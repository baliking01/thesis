\Chapter{Introduction}

This thesis work presents an interpreter library/package for the Fuzzy Behavior Description Language (FBDL) implemented in the Octave programming language.

This initial description of the problem at hand can either be informative or utterly confusing for anyone reading it for the first time, simply because it entails many concepts perhaps still unknown to the reader. But it is quintessential to state the main topic being tackled as to not lose sight of it amidst the following discussions that will eventually lead up to the task itself. Anyone with a moderate to advanced knowledge in the field of programming can easily comprehend the workings of an interpreter if described properly, however simply mentioning that without any prior introduction to the underlying principles would still leave a fairly large gap in the reader's mind concerning the motivation behind such a language and also its use of a myriad of fuzzy logic related ideas. Therefore it is necessary to treat the subject as a whole and describe not only the technical implementations and results, but the theory as well, on which all of it is built.

With this in mind, the work is split into two main sections along with this preceding foreword to allow for some clarifications and provide a greater description of the whole subject matter. The first half exposes the reader to core concepts related to fuzzy logic and incrementally builds them up into its more complex and intricate applications. Also in this part the idea of behavior control and fuzzy state machines are presented along with various mathematical models to help with formalization; it also goes into detail about the specifications of the aforementioned Fuzzy Behavior Description Language. The second half of the work delves into the implementation and inner workings of the interpreter. In the beginning, decisions regarding language specific implementation and other architectural considerations are discussed; possible alternatives are slightly touched upon. Following a general overview of the process of interpreting a language, each stage and their operations are shown separately in detail along with possible corner cases that may require special attention and samples of unit tests to check the integrity and correct operation of the program. Finally the reader is provided with working examples of source code written in FBDL and also a demonstration of said code, where the output of the interpreter can be verified.

The main difficulty lies in connecting the different parts of the overarching subject, so as to allow the reader to indulge in this work without getting lost consider the following short explanation as a guide to the various topics about to be presented.

The motivation behind creating a programming language, be it any kind, is always attributed to the existence of a specific problem it is trying to solve. This could range from low-level hardware management, such as those found in embedded systems, all the way to server-side applications and numerical analysis.

The language (FBDL) appearing in this work has been constructed to serve a particular application of fuzzy logic, namely that of behavior control. For example when a system, due to some event, reacts or behaves in a certain way based on predefined rules dictating its appropriate response. A fundamental property of fuzzy logic, essentially the fact that it is continuous, make it a prime candidate for such a use, since natural systems are hard, sometimes nearly impossible to accurately model with Boolean-logic.

Describing how such a system would operate, the rules it would follow and the actions it would take can be tricky to model with ordinary programming languages. Therefore, an easier alternative was designed with the primary aim of facilitating the ease of use, particularly even if the user happens to lack any kind of previous formal experience in the field of programming. To further simplify the task of using this language it takes another useful property of fuzzy logic that arises from its continuity: we are able to describe the state of a fuzzy variable or in other words the degree to which it satisfies a certain statement with the help of natural language in contrast to using concrete numerical values.

There exits many applications for fuzzy logic and its extensions, some requiring complex calculations and methods that are quite conveniently present in scientific programming languages such as MATLAB and Octave. For this reason an interpreter library in these languages would provide great utility for programs already using fuzzy logic and open new opportunities for those seeking to venture into such areas.

These ideas constitute the majority of the work, so they shall be further examined at length in the following chapters.
