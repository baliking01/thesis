\pagestyle{empty}

\noindent \textbf{\Large Adathordozó használati útmutató}

\vskip 1cm

A mellékelt adathordozón megtalálható a szakdolgozatban szereplő nyelvi fordító könyvtár teljes forráskódja, mely a következő formában van felosztva:

\begin{itemize}
	\item \textbf{thesis}: A dolgozat \LaTeX\ forráskódja, valamint a lefordított \texttt{.pdf} fájl.
	\item \textbf{interpreter}: A program forráskódja, amely modulok szerint szét van szedve még további részekre.
	\item \textbf{images}: Képek a dolgozatban szereplő szabályfelület szemléltetésére.
	\item \textbf{readme.md}: Programhoz tartozó telepítési és felhasználási instrukciók.
\end{itemize}

\noindent Továbbá az interpreter mappában az alábbi felosztások szerepelnek:
\begin{enumerate}
	\item \textbf{lexer}: Lexikális elemzőhöz tartozó függvények.
	\item \textbf{parser}: Parzerhez tartozó függvények.
	\item \textbf{engine}: Viselkedés motorhoz tartozó függvények.
	\item \textbf{simulator.m}: Program belépési pontja. Ezen a függvényen keresztül tudja a felhasználó használni az értelmezőt.
	\item \textbf{surfaceTest.m}: A szakdolgozatban szereplő minta szerinti szabályfelületet ábrázoló teszt program (example.txt-ben lévő FBDL kód alapján).
	\item \textbf{readme.md}: Programről szóló rövid leírás.
	\item \textbf{example.txt}: Egyszerű példakód az FBDL használatának bemutatására.
	\item \textbf{test.txt}: Terjedelmes FBDL kód a fordító teszteléséhez.
\end{enumerate}

A program használatához szükség a GNU/Octave környezet, melybe át kell másolni az interpreter mappát, illetve annak elérhetési útvonalát megadni a környezeti változóknál, hogy fel lehessen hívni a \texttt{simulator.m} függvényt. Az értelmező további függvényinek használata és FBDL programok ``futtatása'' a dolgozat 5. fejezetének végén található leírások szerint történik.