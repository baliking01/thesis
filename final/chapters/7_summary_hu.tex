\pagestyle{empty}

\noindent \textbf{\Large Konklúzió}

\vskip 1cmA szakdolgozatban tárgyalt értelmező Octave-ban készült el és lett implementálva; MATLAB környezetben még nem történt tesztelés, azonban mivel mindkét nyelvet figyelembe véve zajlott a tervezés, így a legrosszabb eshetőségben is csak minimális vagy bizonyos esetben semmilyen módosítást nem kell végezni a kódban. A teljes forráskód megtalálható a fizikai példányhoz mellékelt adathodozón, valamint a GitHub weboldalon is elérhető, mely a \url{https://github.com/baliking01/FBDL\_Interpreter} címről letölthető.

Az értelmező futtatási környezetet biztosít FBDL programok számára, továbbá számos függvénnyel a felhasználó kezelheti és igény szerint befolyásolhatja a fuzzy állapotgép kiindulási értékeit, valamint megváltoztathatja vagy elemezheti annak jelenlegi állapotát. A jövőbeli célokra vonatkozóan, az értelmező következő verzióiban és iterációiban több olyan funkciókat kell hozzáadni, amelyek tovább segítik az említett a folyamatokat. Ezen felül nyilvánvaló, hogy az engine struktúra folytonos adogatása igen körülményes; ezt az osztályok használata jelentős mértékben megkönnyítheti a jelenlegi implementációhoz képest. Azonban ennek ellenére, az osztályok hanyagolása miatt, így kompatibilis marad a program mindkét nyelv korábbi verzióival is. Mindezt figyelembe véve, leszámítva hogy ez egy használható megoldás, a jövőben felhasználóbarátabb megközelítéseket és implementációkat kell kidolgozni, figyelembe véve a teljesítményt és a memóriakorlátokat.

Ezeknek a célnyelveknek nagy előnye a rengeteg elérhető tudományos célű függvény, amelyeket használhatunk a fuzzy állapotgéppel való interakcióhoz; például a szabályfelületek megjelenítése. Mivel az FBDL eredeti célja, hogy könnyen kezelhetővé tegye az etológiai szimulációs modellek programozását, a fejlesztőnek nem szükséges a bonyolult számításokkal és adatstruktúrákkal foglalkoznia azáltal, hogy mindkét nyelv absztrakciókat kínál erre, így a felhasználó teljes mértékben a FBDL forráskódra összpontosíthat.

Bár az itt leírt értelmező teljesen működőképes, még mindig sok dolog van, amit tovább kell fejleszteni és néhány, amit működését illetően módosítani szükséges. A fuzzy logika területén folyamatosan zajlanak a különféle kutatások és ez a program hasznos felületet biztosíthat, valamint segítségére válhat azoknak, akik a fuzzy viselkedés gyakorlati alkalmazásaihoz keresnek egy megfelelő munkakörnyezetet.