\Chapter{Design}

\Section{Requirements and Technical Considerations}
Due to past research around fuzzy logic and its application in behavior control the need for having an adequate language along with a library or interpreter to make calculations easier also arose. For avoiding ambiguity, in the following discussions the words library, program and interpreter are used interchangeably and should be treated as referring to the same concept, namely the FBDL interpreter. The entire operation of the program is based on the preceding works and research in fuzzy behavior control and behavior-based systems \cite{pillerkovacs2019}.

Many of the applied examples were produced in the MATLAB language, therefore facilitating the need for  developing a framework capable of operation in that environment. The similarity between the MATLAB and Octave programming languages presented a great opportunity; implementing the interpreter in such a way as to conform to both languages would make it more accessible and allow for wider usage. Therefore the decision was made to produce a program that uses in its operation only minimal parts that are found in both languages, thus making it compatible and interoperable.

To further elaborate, the usage of language specific components and features should be kept at a minimum or avoided altogether if possible, for example the definition of classes or application of several built in functions, function definitions, unit tests etc. are such areas where the two target languages tend to differ quite majorly. It also does not help that at each iteration and new versions these differences grow ever larger since new features are added and perhaps old ones modified or removed, hence implementation with only basic and fundamental features of each language should be prioritized to the utmost extent.


\Section{Structure of The Interpreter}
The difference in function definitions, the most noticeable one being that in MATLAB scripts cannot contain any local function definitions before version R2016b, and also other incompatibilities related to the syntax of this operation are the main reason that the program is organized in a way so that every function definition occupies its own $.m$ file. 

[Add figure of program hierarchy, tree diagram]

Since the interpreter is not a standalone program, rather a library, its entry point is a function that gets called from the user of this library. From there it goes through each stage of interpreting the input; it is comprised of 3 major parts and a lot of smaller, auxiliary functions that help in completing the task, along with making the program code more readable and modular.

[Add figure of abstract operation, eg.: interpreter parts and data flow]


The program's entry point is a callable function that takes either a string or a file path as input and returns the solution vectors resulting from the fuzzy calculations. In both cases the content supplied must be a valid piece code written in FBDL. 

[Extend with return value and engine/state machine]
\begin{octave}
function retval = main(input, type)
  retval = 0;
  addpath("lexer");
  addpath("lexer/utils");
  addpath("parser");
  addpath("engine");

  content =  "";
  if strcmp(type, "f")
    content = fileread(input);
  elseif strcmp(type, "s")
    content = input;
  else
    % TODO: Print usage
  end

  behavior = parser(content);
  % Fuzzy state machine
  %stateMachine = createStateMachine(behavior);
  %step(stateMachine);
end
\end{octave}

Functions that facilitate the usage of the program must first be ``included'' with path definitions treating the above entry point as root, since they are stored in their separate files. Supplying the function with either type of valid input will initiate its operation, anything other than that will result in an error and the user will be provided a message on basic usage.

The function $parser()$ being called first might be a bit counter intuitive, but the lexer and parser operate simultaneously, not in a procedural manner, where the output of the lexer is the input to the parser. Of course doing it that way is also possible, returning a vector of tokens and then parsing that, however it is not just less efficient, but at the same time takes away the ability to report errors with correct positional messages, since that information is carried by the lexer and not the tokens.

The sequence of operations is perhaps best demonstrated via a simple model:

[Add pseudo code to describe functioning]

The parser continuously requests tokens from the lexer and at the same time it builds the syntax tree for the entire program, after which this completed structure is passed to the engine for calculations.

\Section{Data Structures}
Regarding the data structures used in the program, the most compatible with both languages were found to be the \textit{struct} and \textit{cell}, hence all complex objects are stored in such a manner. For clarity,  the \textit{cell} data type is not used as frequently, but in some parts it is more appropriate than other solutions. The most important of these objects is the \textit{lexer}, but others include \textit{token} and the \textit{syntax tree} along with other minor internal data structures that store and manage the information read during or after lexical analysis.

\SubSection{Lexer}
This structure holds the input and several key information about the position of the cursor currently analyzing the text. Due to organizing every function into its own file the lexer can only be either a global structure that gets modified by any given function or it is created locally in the top most function in terms of calling hierarchy and is passed around by every other function that needs either access to the input stream or the metadata stored in it. A separate function for creating a lexer is provided in foresight to unit testing, so as to avoid having to create it in every single test case.

\begin{octave}
function lexer = createLexer(content)
  keywords = {
    "universe", "rulebase", "end", "description", "rule",...
    "when", "and", "is", "init",  "use"
  };

  lexer = struct(
    "content", content,
    "content_len", length(content),
    "cursor", 1,
    "line", 1,
    "beginning_of_line", 1,
    "token_begins", 1
  );

  lexer.keywords = keywords;
end
\end{octave}

The input stream, or as \textit{content} in the lexer structure, is processed by one character at a time and the \textit{cursor} represents how many characters we have read so far, in other words our current position withing the supplied text. Lines are incremented with each encounter of a \textit{\\n (newline)} character. The staring position of the given token is also stored to allow for copying the value and report error messages, indicating the position of the incorrect token.

\SubSection{Token}
Defined as having the fields: type and value, where the former can be any of the valid token types described by the grammar rules of the FBDL and the latter takes on the actual value read from the input stream. Below are the types accepted and the possible values a token may take:

\begin{itemize}
	\item keyword: \textit{universe, rulebase, init, end, description, rule, when, and, is, use}
	\item string: \textit{Any character sequence between two '' (double quote) symbols}
	\item number: \textit{Any number, be it decimal or whole with . (dot) for separator in case of the former}
	\item identifier: \textbf{RESERVED} \textit{Any ASCII alphanumeric sequence starting with either a letter or \_ (underscore)}
	\item terminal: \textbf{RESERVED} \textit{Special symbols such as (), [], \{\} , ::}
\end{itemize}

\textit{Please note that tokens designated as \textbf{reserved} are either already in the program code, but not actively used or there are room for them should the need arise for future extension.}

Tokens are ``produced'' or emitted by the \textit{emitToken}  function that constructs a token structure with the correct type and value.

\begin{octave}
function token = emitToken (type, value)
  token = struct(
  "type", type,
  "value", value);
end
\end{octave}

\SubSection{Syntax Tree}
[Universe and Rulebase]